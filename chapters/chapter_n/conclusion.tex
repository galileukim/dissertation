\section{Conclusion}

This dissertation provides a set of empirical and theoretical research projects on state capacity. This journey unexpectedly started in the slopes of the Peruvian Andes, as I learned from municipal bureaucrats about the challenges of administering and delivering public services to the population. While the topic of bureaucracy has always puzzled my family and friends -- why would anyone care about this? -- I still hold firmly to the belief that understanding how these bureaucracies operate is an important prerequisite for making ensuring access to high quality public services. I have been privileged to explore this topic in Brazil, guided by the knowledge and warmth of my countrypeople.

The questions and answers explored in this dissertation provides us with a few lessons. Firstly, that while bureaucratic turnover is a well-known feature of developing countries, it may be caused by the very democratic institutions that are the bedrock of a functioning structure of governance. The separation of powers institutionalized between local executive and legislature can have unintended consequences, as executive leaders face the need to coopt legislators to their policy agenda. In the absence of other forms of political currency, patronage may be the cost of democracy.

Additionally, it is important to note that while these democratic institutions have important implications to certain departments in the bureaucracy, it does not affect bureaucrats equally. In particular, high-level bureaucrats are insulated from the political fates of their principals as they may be too valuable to be let go of. As a result, even if elected officials are removed from office, their bureaucratic counterparts remain, exposing local governments to the possibility of recurring corruption. Policies designed to counteract this behavior may prove ineffective as is. However, simple improvements, capitalizing on complementarities between its components may have large payoffs.

Finally, it is clear that partisan networks are pervasive in local bureaucracies. What stands out from data on employment careers is that partisan bureaucrats stem primarily from the wealthy sectors of the municipal economy. Contrary to longstanding view of party machines, sharing the spoils with a wide net of poorer constituencies, the evidence clearly suggests that it is the wealthy who enter through a narrow door into the upper echelons of the local bureaucracy. Moreover, these wealthy partisans gain access to tenured positions, high compensations and have lower educational requirements than their non-partisan counterparts.

What remains to be done? There are a few threads that may be worthwhile exploring for future scholarship. In particular, the nexus between public and private sector employment remains understudied and Brazil, through a rich panel of all formal sector workers, provides a unique opportunity to further explore this revolving door. Do politicians become bureaucrats and vice versa? How porous are the boundaries between public and private sector, beyond party members and high-level bureaucrats? And what are the implications for local governance and public service delivery as a result of these dynamics? These questions have important implications for the quality of services and governance for developing countries as a whole. Much remains to be explored.