
\section*{Acknowledgments}

Science, especially a social one, is a collective endeavor. Fortunately, at Princeton I have not been only able to stand on the shoulder of giants, but learn from them. 

I am grateful to my dissertation commitee for their mentorship and guidance in this journey. Deborah Yashar, my chair, has taught me how to become a scholar, a lifelong lesson I will cherish dearly. Deborah's kind support and constructive insights have led me to intellectual heights I could never reach alone. Matias Iaryczower has taught me how to take ownership and responsibility for my work, and I am immensely grateful for that. From the classroom to coauthorship, I have learned so much from you. Gracias Mati. Guadalupe Tu\~{n}on has generously advised me, in the final stretch, to take a step back, think clearly and walk forward. Working together on these research projects has been a delight. Thank you all.

Friends and colleagues have reminded me constantly that it does not matter how far we go, we can always push forward together. I am grateful to Jos\'{e} Mar\'{i}a Rodriguez Valadez, Will Horne, Dan Gibbs, Beatriz Barros, James Mao, Guilherme Duarte, Elsa Voytas, Michael Kistner, Naoki Egami, and Carolyn Barnett for their friendship and solidarity in our collective journey. Romain Ferrali, one of the coauthors for a chapter in my dissertation, has encouraged me to become a better researcher, developer and professional, all to a jazzy tune. Matheus Hardt has joined forces with me tackling the Brazilian educational data and later, in our data science forays for the Brazilian federal government. Members of the Latin American Politics Workshop, Brazil Lab, Program for Latin American Studies and Political Economy Colloquium have pushed my research agenda with their constructive feedback and curiosity. Thank you.

Fieldwork in Brazil would not have been possible without the generous help of Norman Gall and the team at the Instituto Fernando Braudell. They have received me with open arms, and I have learned tremendously from their generosity. Sandra Leite and the team at SEDUC, UDIME and COPEM have provided me with more help than I could possible acknowledge. To their warmth and countless laughters, obrigado, e um cheiro! Resource and funds for the fieldwork were generously provided by Mamdouha Bobst Center for Peace and Justice, the Program for Latin American Studies and the Princeton Institute for International and Regional Studies.

Lastly, my deepest gratitude to my family -- which has now grown! My parents, Sang Ho Kim and Jung Ah Lee, arrived as immigrants in a foreign land, and not only survived, but thrived. Their love, support and guidance throughout my life has led me to where I stand today. 엄마, 아빠, 감사합니다. Carol has been my anchor and the best little sister a brother could ask. I love you so much. And 누나, thank you for your support and kindness, it's wonderful to see how you, Thomas and Nico keep growing.

Finally, to Kimchi and my lifelong partner, Jill Chien. We have made it through adversity and joy, and in the warmth of your embrace, I know that we can become even better, together: kinder, more curious, and pursuant of our hopes and aspirations. I love you all so much, and thank you for always being here. Let's keep moving forward.
