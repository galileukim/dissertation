\section*{Introduction}

\epigraph{\singlespacing In short, \emph{both} state capacity and politics must be studied if we are to explain state performance -- especially in the developing world.}{\singlespacing Cente\~{n}o, Kohli and Yashar, \emph{States in the Developing World}}

\doublespacing

Citizens across the world rely on governments to provide key public services. Whether it be to administer a vaccine or teach the alphabet, bureaucrats are responsible for fulfilling this government mandate. These public officials span all levels of government: from the frontline, to management and the highest levels of public office, bureaucrats play a crucial role in shaping who receives public services and how these are administered. Under the broad umbrella of state capacity, I sought to understand how to make bureaucracies work better. I focused on why these actors behave the way they do, all the while raising questions regarding the political dynamics in which they are embedded.

In this dissertation, I explore the political administration of bureaucracies. Using Brazilian local governments as a case study, I analyze how political institutions and actors shape well-known empirical phenomena such as patronage and corruption. The goal is to understand how and why bureaucrats choose to behave the way they do, as they navigate the institutions in which they are embedded. In interviews, formal modeling and exploratory data analyses, I illustrate their choices and incentive structures, both individual and institutional, that lead them to act in ways that are often counterproductive to the quality of management of public services. It is my hope that shedding light on how these decisions take place can inform the design of better policies to reshape them.

In the first chapter, I explore how local governance institutions shape patronage dynamics in the educational sector. Combining qualitative fieldwork evidence with large scale administrative data, I argue that patronage is driven by local mayor's need to coopt the local legislature. In particular, I find that, consistent across different model specifications, the stronger support a mayor has in the legislature, the less patronage occurs. A formal model on buying supermajorities provides a theoretical framework to understand this finding. Additionally, I show that patronage appointments have knock-on effects on educational outcomes, with lower test scores in schools exposed to staff turnover.

In the second chapter, coauthored with Romain Ferrali, we focus squarely on bureaucrat's decisions to engage in corruption. We combine randomized anti-corruption audits with structural estimation to identify the conditions under which auditing technologies can reduce corruption in Brazilian local governments. We find that anti-corruption audits have only limited effects on career choices for high-level bureaucrats. In particular, in municipalities found to be corrupt and exposed to the audits there is no evidence of changes in dismissals or departures of bureaucrats. To understand this finding, we model bureaucrats' choice to engage in corruption and apply our model to the data through structural estimation. Counterfactual analysis identifies strong complementarities between the components of the audit, suggesting that pulling all levers may be more effective at reducing bureaucratic corruption.

The last chapter focuses on understanding the breadth of partisan affiliation in local bureaucrats. Joining micro-level data on party membership with employment histories of all formal sector workers, I shed light on the career paths and characteristics of party members across private and public sectors. Career data suggests that party members who enter the bureaucracy are primarily selected from the high-income tail of the distribution in the private sector. Once hired, party members concentrate in the upper levels of management in the public sector, receiving contract tenure, higher wages and staying for longer than non-partisan counterparts. Finally, I provide evidence that patronage can be used to incentivize wealthy donors to contribute to local parties, raising questions regarding party building and finance, as well as representation in developing contexts.

Understanding how to make bureaucracies work better is necessarily an unfinished agenda. The multiplicity of bureaucratic institutions, actors, and government structures simply defy a unique answer. That is our asymptotic frontier. However, I remain convinced that a combination of methods can shed light on these complex institutions. In fact, it may be the only principled way to move forward and approximate us to truths. I hope that the structure of this dissertation reflects this intellectual humility: states need to be unpacked, and meso-level theories that bring together substantive concerns, theoretical modeling and rich data -- both qualitative and quantitative -- can provide us with tentative but grounded answers to pressing questions regarding state capacity. 

Much has been said, and much remains to be done. It is my hope that the projects I have embarked on in my research career at Princeton will leave a trail behind that other scholars can explore. All the data and replication files used for this dissertation are open source and can be found at my GitHub page \href{https://github.com/galileukim}{here}. As I have learned from others, I hope that other scholars can learn from my advances, avoid my shortcomings, and push even further ahead. My journey does not end here though! I recently joined the Bureaucracy Lab at the World Bank, as well as the Knowledge, Innovations and Communications Team at the IDB. I am partnering with the Brazilian National Public Administration Department for research projects on bureaucracies. There is much to be done.

\newpage