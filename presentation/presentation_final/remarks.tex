\documentclass[12pt,a4paper]{article}
\usepackage[utf8]{inputenc}
\usepackage{amsmath}
\usepackage{amsfonts}
\usepackage{amssymb}
\usepackage{lmodern}
\author{Galileu Kim}
\title{Prospectus Defense: Opening Remarks}
\begin{document}
\maketitle
\section{Motivation:}
This project is motivated by the challenges introduced by the decentralization of public services. The idea for this project started when I first began to conduct research on local governments in Peru. At that time, I became familiarized with some of the administrative challenges generated by the delegation of public service provision to local governments. Some of them lacked the expertise necessary to provide healthcare and education for the local population. Others struggled to recruit talented bureaucrats to staff their agencies.

This was not an abstract problem. Nurses discussed with me the challenges of convincing often illiterate pregnant women refuse to attend prenatal care, putting at risk both their lives and of their infants. Parents shared with me their concerns for their children's future, given the deficient state of schools in their municipality (\textit{distrito}). Mayors voiced their frustration at being unable to retain teachers, who at the first opportunity moved to the closest city. Yet not everything was failure.

In the end, public service provision is an administrative problem, that required the coordinated efforts of these street-level bureaucrats and elected officials responsible for recruiting and managing them. When I first entered Princeton, I was eager to learn more about how bureaucracies operated, the intersection between political agency and public service provision, as well as how public administration could be refashioned to provide better public services for communities which seemed so desperately underserved.

Yet I found myself struggling with the macro-level focus of major scholarly works on the bureaucracy. These were national-level institutions, slow-moving and focused on macroeconomic policy or regulatory oversight. Furthermore, I was surprised to see how high the wall of separation between public administration and political decision-making had been built, without ever questioning the assumptions underlying it. Works focusing on executive decisions painted a desolate landscape of corruption, clientelism, and a pre-modern era of personalistic politics.

I struggled to reconcile these analytical frameworks with what I had observed on the ground in Peru. I found myself unwilling to assume that all actions by politicians were done with the intent to demolish democratic institutions through private transfers to voters. I also struggled to accept that the legacy of the past was insurmountable and the tragedy of a flawed genesis condemned institutions to a single path. There had to be a middle-ground between structure and agency that could carve the way forward from a theoretical standpoint.

\section{A middle-ground: politician's dilemma}

Part of the solution was to be found in re-reading carefully Geddes' \textit{Politician's Dilemma} and more contemporary works on democratic accountability by political economists. I began to understand the problem from an electoral point of view, and the set of incentives which motivate politicians to invest in state capacity. In previous iterations, I had just assumed that politicians wanted to do a "good job", providing voters with high quality public services. That, of course, was too simplistic.

Rather, it is more natural to assume that politicians face a trade-off. Engaging in patronage - thus consciously dismantling bureaucratic institutions - must imply a cost for politicians, otherwise everyone would invariably do it, especially in the developing world where sanctioning mechanisms are not well developed. Looking at the empirical record, this seemed to be far from the case. There was clear evidence, at least in Brazil, that reality was more complex than that: some mayors had really chosen to invest in improving their staff, while others did not.

I partially solved this problem by assuming that engaging in patronage has a public service quality cost: the more incompetent bureaucrats you bring in, the worse will be performance. This introduces a real complexity to the problem: some mayors may not care at all about re-election and therefore extract everything in rent. This is the case of a recent governor in Rio, who extracted hundreds of millions of dollars from public coffers through illicit contractual agreements. Most mayors, however, by the Law of Large Numbers should be in between: engage in some patronage, but also invest enough in bureaucratic institutions to appease voters.

Following other political economists, I thought of investments in state capacity as essentially a long-term gamble, with reputation playing a crucial role. Voters are uncertain about what the true intentions of the politicians are, but can vote both prospectively and retrospectively. Learning from what happened in the first period, voters learn who the politicians are, but do so in an imperfect fashion. As a result, there is slack for politicians to engage in some patronage. Voters, furthermore, are social welfare maximizing, a strong assumption that I will relax in the future.

What I am most dissatisfied with is with the cost of engaging in meritocratic hiring. It is not clear to me how to justify the private rents in the model without thinking about them more carefully. In Geddes' account, the tradeoff is between long-term economic development and the immediate redistribution to parties in order to maintain political survival. Here what I am modeling is not particularly clear: is it mere corruption? Graft? Political support? I need to think about this more carefully in order to convince myself of the story the model is telling us.

\section{Research Design}

A crucial next step in my research project dissertation will take place during this summer, where I will finally be able to travel to municipalities in Brazil to study the educational system from the bottom-up. All of my work so far has been in the vein of familiarizing myself with Brazil's educational system, and what has surprised me most is the lack of academic work looking at it from a municipal perspective. More often than not, it is centered at the national level, with little inquiry about what explains municipal variation in educational performance.

I have secured contacts with NGO's in Brazil that have operated at the municipal level, coordinating closely with local governments to implement "reading circles". They operate primarily in the Northeast, a region known for its historically poor educational performance. It is also the site of Judith Tendler's pioneering study on Ceará, one of the poorest states in Brazil that was surprisingly able to outperform even other more developed states in terms of educational performance. I see this as an opportunity to engage in theory building and refinement rather than searching for more data.

There are a set of actors that I have to interview. Mayors are usually difficult to get a hold of, but it is far easier to secure access to teachers and principals. I suspect that secretaries of education are more readily accessible than elected officials and they will clearly play a crucial role in the story. I am also interested in the perspective of the local population with regard to the quality of education their children receive, and how this affects perceptions of the mayors' competence or level of commitment to the local population. 

I am also curious about how the hiring process varies across municipalities, what are the different instruments used to recruit the population into accepting a public sector job as a teacher, attrition rates, and choices by teachers. It will be more of a poke and prod approach, to validate whether the sketches of my theory make sense and how to revise it in the light of qualitative evidence. It is clear to me that I need to model the supply side more rigorously: bureaucrats are not simply floating in the air, waiting to be caught by politicians. Teachers can opt in or out.

This fieldwork will be unfortunately brief, but in doing so I hope to establish the contact network to return to Brazil, in what I presume will be the second semester of fourth year. This fieldwork experience will provide me with a better intuition of whether or not my theory makes any sense, and to better understand how the municipal educational system works from the inside out. Does it make sense to assume that educational systems are reconfigured through executive decisions? These systems are complex and the ways in which they evolve do not depend solely on executive action.

I believe this is the natural progression in my empirical work. Much of my discussion with colleagues and faculty has centered on the importance of describing both the demand and supply side of teachers, and given that I have the entire Brazilian labor market data, there is much that can be done in that direction. Parts of this can be already incorporated now: I can now what is the underlying distribution of works in each municipality, which will let me determine with greater precision what is a "meritocratic hire". It has to be context specific: a good teacher in Sao Paulo is quite different from one in Quixaromibim given sociodemographic differences.

\section{Next Steps}

As stated above, most of the upcoming months will be spent in both conducting fieldwork interviews and analyzing the qualitative evidence I collect there. In terms of quantitative analysis, I will build on a paper I will be circulating for LASA this upcoming conference, where I analyze spatially and over time the evolution of municipal teaching staff. The purpose of the paper was mostly descriptive, but I believe I will be able to incorporate the empirical analysis there to my dissertation.

I also plan on securing funding for more extended fieldwork on municipal education in Brazil. This summer will be exploratory at best, helping me get a sense of how to conduct interviews in Brazil and pinpoint strategies for gathering information on staffing decisions. This will also provide a network from which I can further expand my sample of case studies to leverage variation in explanatory variables. This is a weakness in the prospectus which I need to address: as Hillel pointed out, the audience is interested in the independent variables, not so much on the dependent variables.

In terms of causal inference, I am pinning down exogenous variation in federal laws which will provide instrument for estimating causal effect. In particular, in 1998 the prospect of reelection was introduced for municipal mayors, an institutional change that could dramatically modify the incentives for politicians to invest in state capacity. Additionally, there are different laws concerning coalition formation which vary with population thresholds, which could induce changes in incentives to engage in patronage.

In the end, there is still much to be done, and I feel like I have only begun exploring the richness of the data which I have gathered so far. I plan to refine the model given the qualitative evidence which I gather this summer in Brazil, incorporating explanatory variables that tell us a fuller story of how bureaucratic institutions are modified in municipalities across Brazil. This melding between formal theory, qualitative and quantitative evidence is difficult to achieve. But I hope that this dissertation project will be able to tap into these different strengths and have them complement one another.

\end{document}